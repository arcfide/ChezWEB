\input eplain
 
\font\titlefont = cmr17
\font\chapfont = cmb16

\def\prologue{%
  \singlecolumn
  \centerline{\chapfont Prologue}
  \doublecolumns}
\def\endprologue{\par\vfil\break}
 
\null\vskip 3in
\centerline{\titlefont ChezWEB User's Guide}\bigskip
\centerline{Aaron W. Hsu}\smallskip
\centerline{\it arcfide@sacrideo.us}\medskip
\centerline{\today}
\vfill\noindent
Copyright $\copyright$ 2010 Aaron W. Hsu $<$arcfide@sacrideo.us$>$. All
rights reserved.\medskip\noindent
This document may be redistributed without restriction provided that
it is redistributed without fee and that it remains unaltered and
unchanged from the original. Please report errors to the author.\par
\vfil\break

\doublecolumns

\prologue

Some time ago, the now Professor Emeritus of the Art of Computer
Programming Donald E. Knuth began writing programs. Sometime after
that, he began to construct programs in a new manner. This manner, he
documented and labeled ``Literate Programming.'' In Professor Knuth's
vision, a program is not constructed to be read by the machine, but
rather, to be read as a pleasant book is constructed, to be read by
the human. In this way, one constructs and builds the pieces of a
program together, as you might build up the necessary elements of
Math, surrounding them with exposition, and ordering them in the
manner that best reveals the program's working and meaning to the
reader. 
 
This somewhat radical approach to programming leads to a drastically
different perspective on how to write programs. Indeed, I feel that
writing my programs using Literate Programming has greatly improved my
ability to maintain and improve these same programs, and moreover, to
understand these programs as I am writing them. I enjoy writing and
seeing the results of my writing, both in a printed or screen-readable
form, as well as in a machine executable form. 
 
While I profess no particular skill in either writing or programming,
I do profess to enjoy both. Indeed, this dual enjoyment is a necessary
condition for good programs, and is especially important in Literate
Programming, because it exposes your thoughts in two ways. This
enforced discipline can be embarassing at times, but inevitably leads
to a better programmer. 
 

\endprologue
 
\bye
